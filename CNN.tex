\documentclas[12pt]{article}
\begin{document}


A study on the role of competitive neuron layers in deep learning.

Rakabet§ci n§oron katmanlar§in§in derin §o§grenmedeki rol§un§un incelenmesi. 

 
\section*{Ozet}

Derin §o§grenme~\cite{LeCun:2015} teknikleri b§uy§uk veri setlerinin
s§in§ifland§ir§ilmas§i yolunda §onemli geli§smelere olanak sa§glam§i§st§ir.
S§in§ifland§irma problemi a§g§irl§ikl§i olarak y§onlendirilmi§s §o§grenme
y§ontemleri ile §c§oz§ul§ur. Derin n§oron a§g§in§in a§g§irl§iklar§in§in
olu§sturulmas§i veri ve beklenen sonu§ctan olu§san bir §o§grenme setinin
kullan§ilmas§in§i gerektirir. Projede §onerilen rekabet§c§i n§oron a§glar§i ise
y§onlendirilmis §o§grenme y§ontemi ile §o§grenme ger§cekle§stirmez. Verilen
veri setine g§ore bir a§g§irl§iklar matrisinin olu§smas§i rekabet§ci n§oron
a§glar§inda esast§ir. Rekabet§ci n§oron a§glar§in§in karma§s§ik g§ur§ult§ul§u
verilerden bilgi §uretebiliyor olmas§i derin §o§grenme y§ontemlerine
yapaca§g§i katk§i ile ilgili beklentilerin olu§smas§ina sebep olur.

Projenin amac§ina ulasabilmek i§cin de§gi§sik b§uy§ukl§uklerde rekabet§ci n§oron
a§glar§i olu§sturularak §cok katmanl§i networklar i§cinde farkl§i veri
setleri optimum §c§oz§um§un bulunmas§i sa§glanacakt§ir.


\section*{Amac}

G§un§um§uzde bilgisayar donan§imlar§inda g§or§ulen geli§smenin yan§inda yaz§il§im
alan§inda da §cok §onemli geli§smeler ya§sanmaktad§ir. §Ozellikle, yapay zeka
olarak adland§ir§ilan ve bilgisayarlar§in §o§grenme yetenekleri
kazanmalar§i yolunda yap§ilan §cal§i§smalar son on y§ilda yanl§iz akademik
alanda de§gil, g§unl§uk ya§samda da hayat§im§iz§in bir par§cas§i olmu§s
durumdad§ir.  §Ozellikle son on y§il i§cinde derin
§o§grenme olarak da adland§ir§ilan §cok katmanl§i yapay n§oron a§glar§in§in
bilgisayar §o§grenme kapasitelerine yapt§i§g§i katk§ilar konunun
pop§ulerli§gini artt§irm§i§st§ir.

B§ut§un bu ya§sanan b§uy§uk geli§smelere ra§gmen halen bilgisayarlar§in
§o§grenme yetenekleri kazanmalar§i y§on§unde ger§cekle§sen §cal§i§smalar
ba§sang§i§c a§samalar§indad§ir. Bilgisayarlar §o§grenmesin kullan§ilan veri
setlerine ba§gaml§il§i§g§i, veri setlerinde mevcut karma§s§ikl§i§g§in etkleri
gibi konular halen ara§st§irmaya a§c§ik konulard§ir.
  

Projenin amac§i rekabet§ci (competitive) networklar§in derin yapay n§oron
a§glar§in§in performans§ina etkilerinin incelenmesidir. Rekabet§ci-n§oron
a§glar§inda, girdi verisi §uzerinde n§oronlar aras§i y§onlendirilmemi§s
etkile§smeleri n§oron a§g§in§in a§g§irl§ik matrislerinin, girdi verisi setinin
§ozellikleri ile ili§skili belirgin yap§ilar almas§ina neden olur~\cite{Rumelhart:1985}.  Derin
§o§grenme a§glar§inda rekabet§ci a§glar y§ontemi ile elde edilmis
a§g§irl§iklar§in etkisi ile ilgili §cok az §cal§i§sma bulunmaktad§ir.

\section*{§Onem}

Yapay zeka cok say§ida ve farkl§i bilgi, teknik ve teknolojileri i§cinde
bar§ind§iran disiplinler aras§i bir bilim dal§id§ir. Makina-§ogrenmesi,
yapay zekanin temel §ogelerinden biri ve gunumuzde modern hayat§in bir
parcasi olarak gunluk yasamin vaz gecilemezlerindendir. Gunumuzde, cep
telefonlarindan, fotograf makinalarina, oyuncaklardan, sosyal aglara
ve elektronik alisverise kadar gunluk hayatin parcasi olan
makinaogrenmesi teknikleri son yillarda derin ogrenme tekniklerinin
basarisi ile farkli bir boyut kazanmistir.  Derin ogrenme, konusma
algilama, konusmanin yaziya dokulmesi, tercume, resim sekil ve
multi-medya verilerinden tanimlama yapma bilgi uretme gibi pek cok
alanda yapay zeka alaninda gelismelerin oncu teknolojisini
olusturmaktadir. Derin ogrenme tekniklerinin uygulandigi tum alanlarda
basariyi etkileyen, verinin gurultulu olmasi nedeni ile yanilma
oranidir. Planlanan projede derin ogrenme teknolojisinin basarisini
etkileyan gurultulu bilgiden daha saglikli sonuc uretebilme
yontemlerinin arastirilmasi amaclanmistir.

Planlanan calismanin bir onemi de derin ogrenme alaninda yapilan
calismalarin cok yogun olmasina karsin uygulamalar arastirma ve
gelistirme faaliyetlerinin odagini olusturur. Tasarlanan calisma ise
derin noron aglarinda verimliligin arttirilmasi yonunde bir temel
calisma olma ozelligi ile onem acisindan farkli bir yerde
bulunmaktadir.

\section{Materyal ve Yontem}

Gerceklestirilecek cok katmanli noron agilari iki farkli yapiya sahip
olacaklardir. Ilk grup cok katmanli noron yapisi farkli rekabetci
noron katmanlarindan olusacaktir. Bu yapi her katmanin yonlendirilmeis
ogrenme yaptigi bir yapidir. Ikinci gropta ise iki farkli yapinin
birlikte kullanilmsi amaclanmistir. Bu modelde giris katmanlari
rekabetci noron aglarindan olusur. Giris katmanlarindan elde edilen
bilgi, cok katmanli (convolusional) ikinci bir noron aginin girdisi
olarak kullanilir.


Onerilen noron aglarinin basarilari testi icin farkli yapilar ve
parametre setleri secilen veri setleri uzerinde calistirilarak basari
oranlari karsilastirilacaktir. Farkli yapilar ve parametre setleri
arasinda ogrenme hizi, noron sayisi, guncelleme kurallari oldugu gibi
bilincli veri seti~\cite{DeSieno:1988} gibi karmasik tekniklerin
kullanilmasi da planlanmaktadir. Farkli veri setlerinden elde edilen
sonuclarin standart derin ogrenme yontemleri ile
karsilastirilastirilmasi sonucu elde edilecek oranlar basari kriterini
belirleyecektir.

Proje, gunumuzde derin ogrenme teknikleri icin yaygin olarak
kullanilan Tensorflow progami uzerinde insa edilecektir. Google
firmasi tarafindan acik kaynak kodu olarak yayinlanan Tensorflow
programi~\cite{Tensorflow} derin ogrenme alaninda kullanilan
programlar arasinad en populer olanlardan biridir. Tensorlow
programinin en onemli ozelliklerinen biri de CPU ve GPU kullanimina
olanak vermesidir. Ozellikle derin ogrenme tekniklerinde veri ve
noronlar arasi bag matrisinin buyuklugu yuksek persormansli hesaplama
tekniklerine ve donanimina gereksinimi arttiran
ozelliklerdir. Programlarin paralelestirilebilmesi de Tensorflow
programinin bir diger avantaji olarak gosterilebilir.

\section{Arastirma Olanaklari}


Halen 2013 yilinda ``Fikir Dinamiklerinde Dis etkilerin Bilgisayar
Simulasyon teknikleri ile incelenmesi'' baslikli proje (BAP Proje no:
13H4343002) cercevesinde alinmis olan bir adet Zenon islemcili is
istasyonu suren calismalarda kullanilmaktadir.


\section{Gider Gerekcesi}

Halen calismalarin surduruldugu bir is istasyonu bu projede de
kullanilacaktir. Is istasyonu, 2013 yilinda Ankara Universitesi BAP
(Proje no: 13H4343002) destegi ile alinmis olup, gunumuz hesaplama
gereksinimleri icin yeterli olmamaktadir. Sunulan projenin gelecekte
daha kapsamli calismalar icin on hazirlik niteliginde olmasi nedeni
ile hesaplama hacmi goz onune alindiginda halen mevcut is istasyonunun
Tensorflow programinin calismasina olanak verecek CUDA islemcili en az bir
adet guclu GPU ile desteklenmesi zorunludur.

Calismalarin surekliliginin saglanabilmesi icin guclu bir CPU yaninda
CUDA islemcili goruntu kartina sahip tasinabilir bir bilgisayarin
proje cercevesinde kullanilmasi gerekmektedir.


\section{ Calisma Plani} 



\section{B Plani}, 

Planlanan calisma, derin noron aglarinda, hasarli, gurultulu ve
yetersiz veriler ile bilgi uretmek konusunda baslatilan calismalara
bir alt yapi olusturacaktir. Projenin derin ogrenme konusunda ogrenme
basari oranini arttirma yonunde onemli katkisinin olacagi gorusunu
savunmakla birlikte istenen basariya ulasilamamasi durumunda iki
farkli yan calismanin da surdurulecek olmasi projenin verimliligi
acisindan onemlidir.

a) Derin noron aglarinin diger yonlendirilmis ogrenme modelleri ile en
temel farki ogrenme sirasinda disaridan ogretilen nesnelerin
karakteristik (features) ozelliklerinin tanimlanma gereginin
olmamasidir. Bir on katman tarafindan goruntu karakteristikleri ile
ilgili bilginin etkisinin incelenmesi projenin bir yan hedefidir.
 
b) Ogrenci projeleri kapsaminda tamamlanan olan, ``Yapay zeka temelli
kendiliginden karar verebilen dinamik sistem tasarimi'' 
( 2016, BAP Proje No: 16O044303) projesinin uygulama alanlari
genisletilerek goruntu algilama ve gurultulu goruntuden resim
netlestirme konusunda basarili sonuclar elde edilebilecektir.


\section{Konudaki Yerli ve Yabanci Calismalar}

Derin ogrenme~\cite{Hinton:2006,LeCun:2013} teknolojik ve ekonomik onemi
acisindan son yillarin en populer konularindan biridir. Ozellikle,
Google, Facebook, Amazon, IBM, Apple, Samsung gibi Dunyanin onde gelen
teknoloji firmalarinin cok buyuk destekleri ile bilimsel faaliyet ve
arastirma sayisinda buyuk bir patlama yasanan bu alanda tam ve dogru
calisma listesinin olusturulmasi son derece guctur. Bu nedenle konu
ile en yakindan ilgili bazi makaleler ile derin ogrenme teknolojisinin
gelisme sinin temel taslari olan referanslar listelenmistir.

Gunumuze kadar makina ogrenmesi algoritmalari ve modelleri uzmanlarin
kisisel cabalari ile probleme uygun olarak deneme-yanilma yolu ile
tasarlanmaktadir.  Meta-ogrenme algoritmalarin ve modellerin makina
ogrenmesi yontemleri ile tasarlanmasidir. Bu yaklasimda model kendi
kendine ogrenmeyi ogrenir duruma gelir.  Canli beyni ogrenmeye hazir
sekilde yaratilmistir.gerekli bilgilerin ogrenilmesi, depolanmasi ve
hatirlanmasi icin gerekli noron agi yapisi kendiiginden olusur. Bu
yapi ayni zamanda cok katmanli ve her karmanda farkli islemlerden
gecen bilginin islenmesine olanak verecek bir yapidir. Benzer degisimi
kendiliginden gerceklestirebilen noron yapilarinin tasarimi
calismalari gunumuzun en uc makina ogrenmesi problemlerinin basinda
gelmektedir~\cite{Hanxio:2018}.

Ikinci bir onemli konu insan beyninin cok sinirli sayida ogrenme
orneginden yararlanarak ogrenebilmesine karsin makina ogrenmesinde cok
buyuk miktarlarda ogrenme verisi gereklidir. Gunumuz derin ogrenme
yapilari bir cismi bir veya iki kez gordukleri dyrumlarda ogrenme
gerceklesemez. Son yillarda model-tabanli, olcut-tabanli ve
optimizasyon-tabanli meta-ogrenme yontemleri yolu ile makina
ogrenmesinin sinirli sayida ornek kullanilarak gerceklestirilebilmesi
cabalari da onem kazanmistir~\cite{Vinyals:2016}.


Insan beyni duyu organlarindan gelen bilgiyi suzerek bilgi uretmek
uzere hazirlar, yukarida bahsi gecen yaklasimlara ek olarak farkli
katmanlarda islenen bilginin ogrenmede kullanilmasi noron
sistemlerinde ilgi alanina
girmistir~\cite{Rumelhart:1985,DeSieno:1988}.  Rekabetci noron
aglarinda ogrenme sureci icin karakteristik ozelliklerin belirlenmesi
gerceklestirilmistir~\cite{Rumelhart:1985}.  Gunumuzde karsilasilan
problemlerin guclugu ve yeni teknolojiler isigi altinda farki
orgulerin farkli bilgiler ureterek birbirini destekleyici katkilari
ile daha az egitim gereksinimi ile daha optimize ogrenmenin
gerceklestirilmesi proje cercevesinde planlanan calisma olacaktir.


\section{Kaynak Listesi}




\section{Onceki Calismalar}





\begin{thebibliography}{99}
\bibitem{Hinton:2006} G. Hinton, S. Osindero, and Y.-W. Teh. A fast learning
algorithm for deep belief nets. Neural Computation, 18(7):1527–1554, 2006.
  
\bibitem{LeCun:2015} Yann LeCun, Yoshua Bengio and Geoffrey Hinton
Deep learning -Review, Nature 521, 436 (2015)

\bibitem{Hanxio:2018} Hanxiao Liu, Karen Simonyan, Oriol Vinyals, Chrisantha Fernando, Koray Kavukcuoglu, Hierarchical Representations For Efficient Architecture Search, Published as a conference paper at ICLR 2018 

\bibitem{Vinyals:2016} O. Vinyals,C. Blundell,K. Kavukcuoglu,T. Lillicrap,D. Wierstra, Matching Networks for One Shot Learning,arXiv:1606.04080v2.

\bibitem{DeSieno:1988} DeSieno. Adding a conscience to competitive learning.
IEEE 1988 International Conference on Neural Networks, page 117, 1988.

\bibitem{Ranzato:1985} M. Ranzato, C. Poultney, S. Chopra, and Y. LeCun. Efficient learning of sparse representations with an energy-based model. In Advances in Neural Information Processing Systems, number Advances in Neural Information Processing Systems 19 - Proceedings of the 2006 Conference, pages 1137–1144, 2007.

\bibitem{Rumelhart:1985} D. Rumelhart and D. Zipser. Feature discovery by competitive learning. Cognitive Science, 9(1):75–112, 1985.

\bibitem{Tensorflow} Mart\'i n Abadi, Ashish Agarwal, 
Paul Barham, Eugene Brevdo, Zhifeng Chen, Craig Citro, 
Greg S. Corrado, Andy Davis, Jeffrey Dean, Matthieu Devin, 
Sanjay Ghemawat, Ian Goodfellow, Andrew Harp, Geoffrey Irving, 
Michael Isard, Rafal Jozefowicz, Yangqing Jia, Lukasz Kaiser, 
Manjunath Kudlur, Josh Levenberg, Dan Man\'e, Mike Schuster,
Rajat Monga, Sherry Moore, Derek Murray, Chris Olah, Jonathon Shlens,
Benoit Steiner, Ilya Sutskever, Kunal Talwar, Paul Tucker,
Vincent Vanhoucke, Vijay Vasudevan, Fernanda Vi\'egas,
Oriol Vinyals, Pete Warden, Martin Wattenberg, Martin Wicke,
Yuan Yu, and Xiaoqiang Zheng.

TensorFlow: Large-scale machine learning on heterogeneous systems,

2015. Software available from tensorflow.org.  



\end{thebibliography}

\end{document}


